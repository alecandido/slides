\documentclass[9pt]{beamer}

\geometry{paperwidth=160mm,paperheight=120mm}

\usetheme[{titleformat plain}=smallcaps,
          titleformat title=smallcaps,
          titleformat subtitle=regular,
          titleformat section=smallcaps,
          titleformat frame=smallcaps,
          numbering=fraction
          ]{metropolis}
\usepackage{appendixnumberbeamer}

\usepackage{booktabs}
\usepackage[scale=2]{ccicons}


\usepackage{style/defs}

\usepackage{tikz}
\usetikzlibrary{shapes,arrows}
\usepackage{amsmath, bm}
\usepackage{siunitx}
\usepackage{physics,hepnames}
\usepackage{mathtools}
\usepackage{enumitem}
\setenumerate[1]{%
      label=\protect\usebeamerfont{enumerate item}%
      \protect\usebeamercolor[fg]{enumerate item}%
      \insertenumlabel.}
\setitemize{label=\usebeamerfont*{itemize item}%
    \usebeamercolor[fg]{itemize item}
      \usebeamertemplate{itemize item}}

\usepackage{subfig}
\usepackage{colortbl}
\usepackage{multirow}
\usepackage{pifont}

\usepackage{pgfplots}
\usepgfplotslibrary{dateplot}

\usepackage{xspace}
\usepackage{soul}
\newcommand{\themename}{\textbf{\textsc{metropolis}}\xspace}

\DeclareMathOperator{\msbar}{\overline{MS}}
\DeclareMathOperator{\dis}{DIS}
\DeclareMathOperator{\phys}{PHYS}
\DeclareMathOperator{\pos}{POS}
\DeclareMathOperator{\dpos}{DPOS}
\DeclareMathOperator{\mpos}{MPOS}

\title{Towards N3LO}
\subtitle{via NNLO and NLO}
\date{September, 2021}
\author{Alessandro Candido
}
%\institute{N3PDF}
\titlegraphic{
%        \raisebox{5pt}[0pt][0pt]{\includegraphics[height=0.8cm]{../_logos/nnpdf_logo.pdf}}\hspace*{10pt}
        \hfill
        \raisebox{5pt}[0pt][0pt]{\includegraphics[height=0.8cm]{../_logos/n3pdf_logo.pdf}}\hspace*{10pt}
        \includegraphics[height=1.3cm]{../_logos/erc_logo1.png}

        \vfill\vspace*{230pt}
        \includegraphics[height=1cm]{../_logos/unimi_logo.png}\hfill
        \includegraphics[height=1cm]{../_logos/infn_logo.png}\\
        \vspace*{5pt}
        {\fontsize{3pt}{3.5pt}\selectfont
             \begin{center}
                 This project has received funding from the European Union's Horizon 2020 research and innovation programme under grant agreement No 740006\quad \includegraphics[height=5pt]{../_logos/eu-flag.jpg}
         \end{center}}
}

\begin{document}

\maketitle

\begin{frame}{DIS coefficients}

    \vspace*{25pt}

    \begin{columns}
        \begin{column}{0.5\textwidth}
            \begin{table}[h!]
                \centering
                \begin{tabular}{c | c c c } 
                    NLO & light & heavy & intrinsic\\
                    \hline
                    NC & \cellcolor{green!25}\checkmark & \cellcolor{green!25}\checkmark & \cellcolor{green!25}\checkmark\\
                    CC & \cellcolor{green!25}\checkmark & \cellcolor{green!25}\checkmark & \cellcolor{blue!25}\checkmark\footnotemark[1]\\
                    NNLO & & &\\
                    \hline
                    NC & \cellcolor{green!25}\checkmark & \cellcolor{blue!25}partially tabulated\footnotemark[2] & \cellcolor{red!25}\ding{55}\footnotemark[4]\\
                    CC & \cellcolor{green!25}\checkmark & \cellcolor{yellow!25}tabulated\footnotemark[3] & \cellcolor{red!25}\ding{55}\footnotemark[4]\\
                    N3LO & & &\\
                    \hline
                    NC & \cellcolor{yellow!25}\checkmark &  & \\
                    CC & \cellcolor{yellow!25}\checkmark &  & \\
                \end{tabular}
            \end{table}
            + FONLL (cf. \textit{matching conditions})
        \end{column}
        \begin{column}{0.5\textwidth}
            \begin{enumerate}
                \item formerly NC coefficients were misused for this (\textbf{Kirill} computed the correct ones)
                \item coming from \textbf{Felix} thesis with more analytic components and improved interpolation
                \item formerly external to APFEL and fixed setup, still expected from \textbf{Jun Gao}
                \item missing, going to be computed by \textbf{Kirill}
            \end{enumerate}
        \end{column}
    \end{columns}

    \vspace*{10pt}
    There is even another couple of levels of nesting: projections and channels.

    \textbf{Projections} consist of $F_2$, $F_L$, and $F_3$, and they are usually
    computed and published together (even though it is relevant that this is
    not the case for N3LO light).

    \textbf{Channels} result in different expressions (non-singlet, singlet,
    gluon), but they are almost always published together.

    \vspace*{15pt}
    {
        \footnotesize
        \begin{tabular}{c c c c c} 
            \cellcolor{green!25}available & \cellcolor{blue!25}updated &\cellcolor{yellow!25}yet not implemented &\cellcolor{red!25}missing & not planned\\
            \hline
        \end{tabular}
    }
\end{frame}

\begin{frame}{Anomalous Dimensions}
    Most of them were already available and used in APFEL, and they have been implemented in EKO:
    \begin{table}[h!]
        \centering
        \begin{tabular}{c c c c} 
            LO & NLO & NNLO & N3LO\\
            \hline
            \cellcolor{green!25}\checkmark & \cellcolor{green!25}\checkmark & \cellcolor{green!25}\checkmark & \cellcolor{red!25}\ding{55}\\
        \end{tabular}
    \end{table}
    
    We are just missing the N3LO expressions, required only for N3LO evolution
    (\textit{expected} to receive an approximation by \textbf{J. Vermaseren}
    and \textbf{A. Vogt}).

    \vspace*{15pt}
    There is \textit{no need for any extra modification} of EKO at N3LO, we
    just deserve to plug in the expressions.
\end{frame}

\begin{frame}{Scale Variations}
    Scale variations are implemented in \texttt{yadism} (\textbf{scheme C}) as completely
    factorized w.r.t. coefficients functions (instead of inlined).

    The actual ingredients needed to combine them are:
    \begin{itemize}
        \item splitting functions ($x$-space), one order less (NLO DIS requires LO $P$)
        \item beta function coefficients, two order less (NNLO DIS requires $\beta_0$)
        \item combination rules
    \end{itemize}

    \begin{table}[h!]
        \centering
        \begin{tabular}{c c c c c} 
             & NLO & NNLO & N3LO\\
            \hline
            $P^{(n-1)}$ &  \cellcolor{green!25}\checkmark & \cellcolor{green!25}\checkmark & \cellcolor{yellow!25}\checkmark\\
            $\beta^{n-2}$ &  & \cellcolor{green!25}\checkmark & \cellcolor{green!25}\checkmark\\
            rules & \cellcolor{green!25}\checkmark & \cellcolor{green!25}\checkmark & \cellcolor{yellow!25}\checkmark\\
            \hline
            \texttt{yadism} & \cellcolor{green!25}\checkmark & \cellcolor{green!25}\checkmark & \cellcolor{yellow!25}
        \end{tabular}
    \end{table}

    \vspace*{10pt}
    Actually there is one further step: to save time and precision \textbf{convolution}
    of splitting functions are \textbf{inlined}.

    This means they have to be computed analytically, and this requires some
    manual work.

    \metroset{block=fill}

    \begin{block}{\texttt{eko}}
        \textbf{scheme A} \checkmark is implemented, while \textbf{scheme B} \ding{55} is work-in-progress.
    \end{block}
\end{frame}

\begin{frame}{Matching Conditions}
definition\footnote{simplified version, full truth in \alert{\href{https://eko.readthedocs.io/en/latest/theory/Matching.html}{eko docs}}}:
    \begin{columns}
        \begin{column}{0.1\textwidth}\end{column}
        \begin{column}{0.4\textwidth}
            \begin{align*}
                \begin{pmatrix}
                    g \\ \Sigma \\ h^+            
                \end{pmatrix}
                &= 
                \begin{pmatrix}
                    gg & gq & gh \\
                    qg & ns_+ + ps & qh \\
                    hg & hq & hh
                \end{pmatrix}
                \begin{pmatrix}
                    g \\ \Sigma \\ h^+            
                \end{pmatrix}
                \\
                T_n &= ns_+ T_n \qquad\qquad  
            \end{align*}
        \end{column}
        \begin{column}{0.4\textwidth}
            \begin{align*}
                \begin{pmatrix}
                    V \\ h^-
                \end{pmatrix}
                &= 
                \begin{pmatrix}
                    ns_v & qh \\
                    hq & hh 
                \end{pmatrix}
                \begin{pmatrix}
                    V \\ h^-
                \end{pmatrix}
                \\
                V_n &= ns_- V_n
            \end{align*}
        \end{column}
        \begin{column}{0.1\textwidth}\end{column}
    \end{columns}

    status:
    \begin{table}[h!]
        \centering
        \begin{tabular}{c c c c c c c c c c c c c} 
             & nsv & ns- & ns+ & ps & qg & gq & gg & hq & hg & qh & gh & hh \\
            \hline
            NLO & \multicolumn{1}{|c}{} &  & \multicolumn{1}{c|}{} & & & & \cellcolor{blue!25}$\mu$ & & \cellcolor{blue!25}$\mu$ & & \cellcolor{blue!25}\checkmark+$\mu$\footnotemark[2] & \cellcolor{blue!25}\checkmark+$\mu$\footnotemark[2] \\
            \cline{2-4}
            NNLO & \multicolumn{3}{|c|}{\cellcolor{green!65!blue!25}\checkmark+$\mu$} & & \cellcolor{green!65!blue!25}\checkmark+$\mu$ & \cellcolor{green!65!blue!25}\checkmark+$\mu$ & \cellcolor{green!65!blue!25}\checkmark+$\mu$ & \cellcolor{green!65!blue!25}\checkmark+$\mu$\footnotemark[1] & \cellcolor{green!65!blue!25}\checkmark+$\mu$ & \cellcolor{red!25}\ding{55}\footnotemark[3] & \cellcolor{red!25}\ding{55}\footnotemark[3] & \cellcolor{red!25}\ding{55}\footnotemark[3] \\
            \cline{2-4}
            N3LO & \multicolumn{3}{|c|}{\cellcolor{yellow!25}\checkmark+$\mu$} & \cellcolor{yellow!25}\checkmark+$\mu$ & \cellcolor{yellow!25}\checkmark+$\mu$ & \cellcolor{yellow!25}\checkmark+$\mu$ & \cellcolor{yellow!65!red!25}$\sim$ & \cellcolor{yellow!25}\checkmark+$\mu$ & \cellcolor{yellow!65!red!25}$\sim$\\
            \cline{2-4} \\
        \end{tabular}
    \end{table}

    $\mu$ stands for the extra contribution given from taking a threshold
    different from quark mass, and it is proportional to $\log(\mu^2/m^2)$.

    \begin{enumerate}
        \item actually $\text{hq}_\text{ps}$
        \item taken from \textbf{NNPDF}, computed from \textbf{Kretzer \& Schienbein}
        \item \textbf{Kirill} about to compute
    \end{enumerate}
\end{frame}

\begin{frame}[standout]
    Let's discuss...
\end{frame}

\appendix

\end{document}
